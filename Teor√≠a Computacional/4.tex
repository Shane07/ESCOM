\documentclass{article}

\usepackage[spanish]{babel}
\usepackage[utf8]{inputenc} % Para usar tildes Unicode
\usepackage[top=2cm, bottom=2cm, left=2cm, right=2cm]{geometry}
\usepackage[colorlinks=true, urlcolor=blue]{hyperref}

\setlength{\parindent}{1cm}

\title{Forma y contenido en las Ciencias de la Computación}
\author{Ortega Ortu\~no Eder}
\date{} % Para evitar que salga la fecha al llamar a 'maketitle'

\begin{document}
	\pagenumbering{gobble} % Ocultar pagenumber
	\maketitle
	\normalsize{
El escrito trata sobre la excesiva preocupación sobre el formalismo (ver las cosas desde su esencia) de la computación que impide un adecuado desarrollo de las ciencias de la computación dentro de tres áreas importantes: teoría de la computación, lenguajes de programación y educación.

\section*{Teoría de la Computación}
Está claro que para formular una teoría es imprescindible conocer demasiado sobre lo básico del fenómeno de algún tema; simplemente no conocemos lo suficiente acerca de ello (teoría computacional) como para explicarla de una manera abstracta. De hecho tratamos de enseñarla con ejemplos claros que entendemos a detalle y deseamos que eso sea suficientemente claros para así dar idea de principios más generales.
\\

Y sin embargo no es suficiente, ya que tenemos malas concepciones -por ejemplo- acerca de los intercambios de tiempo y memoria, complejidades de un programa, circuitos y otros tantos conceptos de la computación.

\section*{Lenguajes de programación}
La problemática también aplica al campo de la programación y los compiladores, incluso porque puede parecer que en esta área la forma debería ser la preocupación principal. Sin embargo, hay que considerar dos cosas: que los lenguajes se están haciendo con mucha sintaxis y que los lenguajes están siendo descritos con mucha sintaxis.
\\

Los compiladores no son preocupaciones dentro de los significados de expresiones, afirmaciones y descripciones porque el uso de gramáticas libres de contexto para describir fragmentos de lenguajes llevaron a grandes avances en cuanto a la uniformidad, lo que implica tanto en especificación y en implementación.

\section*{Educación}
La educación es otra área donde se confunde forma y contenido en lo que respecta a ciencias de la computación, pero en estos casos la confusión preocupa a un nivel profesionista; esto implica una compleja responsabilidad mayor para elaborar y comunicar modelos de la educación misma.
}

\vspace{2cm}

\section*{Bibliograf\'ia}

\noindent \url{http://web.media.mit.edu/~minsky/papers/TuringLecture/TuringLecture.html}
\\

\large{\hfill \textbf{Hecho en } \LaTeX - \url{multiaportes.com}}

\end{document}