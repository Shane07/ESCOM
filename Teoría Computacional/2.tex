\documentclass{article}

\usepackage[spanish]{babel}
\usepackage[utf8]{inputenc} % Para usar tildes Unicode
\usepackage[legalpaper, top=2cm, bottom=2cm, left=2cm, right=2cm]{geometry}
\usepackage[colorlinks=true, urlcolor=blue]{hyperref}

\setlength{\parindent}{1cm}

\title{Problemáticas en las búsquedas de información \\por Internet con fines académicos}
\author{Ortega Ortu\~no Eder}
\date{} % Para evitar que salga la fecha al llamar a 'maketitle'

\begin{document}
	\pagenumbering{gobble} % Ocultar pagenumber
	\maketitle
	\normalsize{
En el artículo se habla sobre los problemas que surgen al momento de consultar información de la Web con motivos académicos, lo que implica dificultades para los estudiantes cuando están buscando información y seleccionando las fuentes; aquí se destaca cuando y cómo los usuarios creen que toda la información encontrada es verídica y confiable. Todo lo anterior se involucra a un estudio hecho a algunos alumnos de la Escuela Superior de Cómputo (ESCOM - IPN).
\\

El estudio hecho se basa desde el punto de vista de los profesores y su experiencia al clasificar y detectar los posibles orígenes de la información previamente consultada por sus alumnos, siendo en casi todos los casos la presencia de información existente en la Web involucrando las fuentes consultadas, sus características y principalmente la calidad en lo presentado, que finalmente refleja el tratamiento adecuado de los datos.
\\

Este problema se debe a la amplia accesibilidad que permite la Web a la información, aunque el eventual aumento de información también implica una mayor complejidad a la hora de consultar información confiable. La enorme popularidad de este método de búsqueda se debe a su mínimo empleo de esfuerzo y tiempo, a cambio de encontrar información poco confiable.
\\

Hay muchos modelos y sugerencias para buscar información así como para su manejo, lo que provoca debates entre profesores sobre desarrollar habilidades adicionales en los estudiantes o seguir los mismos parámetros para consultar información en medios tradicionales como libros, revistas y demás material informativo.
\\

La consulta de datos en la World Wide Web casi siempre comienza utilizando algún buscador como Google, Bing, Yahoo o cualquier otro, a los que se les indican las palabras clave de la investigación que deseamos realizar; cabe destacar que a palabras clave más generales la información tiende a volverse menos precisa, razón por la que se necesitan recursos adicionales para filtrar la información confiable de la no confiable.
\\

De lo anterior surgen las búsquedas avanzadas para realizar consultas muy específicas, como por ejemplo consultar únicamente ciertos recursos multimedia como ebooks, vídeos, presentaciones, infografías, imágenes, etc.
\\

Otras opciones menos conocidas y más complejas son las opciones de buscar en la Hidden Web o utilizar motores de búsqueda específicos, que incluso podemos configurarlos para dejarlos rastreando sitios y al final enviarnos resultados por correo electrónico.
\\

Uno de los aspectos más importantes de todo esto es sobre evaluar la disponibilidad y calidad de la información encontrada, derivando de las circunstancias que rodeen al estudiante como su interés y conocimiento en el tema; por lógica se puede deducir que alguien muy interesado e informado del tema tendrá mayor habilidad de detectar información confiable de aquél que solamente busca información para cumplir con su tarea y no tener posteriores problemas de evaluación escolar.
\\

Algunos parámetros que pueden ayudar a definir la fiabilidad de la información son identificar al responsable del sitio y creador del contenido consultado, junto a su respectiva bibliografía; otros detalles son la coherencia y ordenamiento de la información, así como su presentación y estructura, es decir que sea facil de leer y entender.
\\

Finalmente otro detalle que va en relación a esto es el plagio de información, que comienza comunmente cuando un estudiante copia y pega información para entregarla a sus profesores, y que se incrementa cuando miembros de instituciones de alto nivel académico en México hacen lo mismo ya que iría en contra de su código ético que rige su misión y visión de formar profesionales preparados académicamente.
\\

Como detalle final, se hizo una encuesta a estudiantes de la ESCOM sobre su forma de consultar información en la Web, la cual fue analizada de acuerdo a rangos académicos (y por consecuencia de edad), en la que los estudiantes de los últimos niveles y próximos a finalizar su carrera de Ingeniería en Sistemas Computacionales se comportaban supuestamente más críticos en el filtrado de información, cuando en situaciones normales los profesores que redactaron el artículo dan a entender que posiblemente los encuestados difieren en los resultados debido a que son constantemente indicados de las deficiencias en sus búsquedas de información.
}

\vspace{1cm}

\section*{Bibliograf\'ia}

\noindent \url{http://eprints.uwe.ac.uk/14120/}
\\

\large{\hfill \textbf{Hecho en } \LaTeX - \url{multiaportes.com}}

\end{document}