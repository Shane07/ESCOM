\documentclass{article}

\usepackage[spanish]{babel}
\usepackage[utf8]{inputenc} % Para usar tildes Unicode
\usepackage[top=2cm, bottom=2cm, left=2cm, right=2cm]{geometry}
\usepackage[colorlinks=true, urlcolor=blue]{hyperref}

\setlength{\parindent}{1cm}

\title{Científicos de la computación construyen \\Supercolisionador de Autómatas Celulares}
\author{Ortega Ortu\~no Eder}
\date{} % Para evitar que salga la fecha al llamar a 'maketitle'

\begin{document}
	\pagenumbering{gobble} % Ocultar pagenumber
	\maketitle
	\normalsize{
Un reciente descubrimiento en el Juego de la Vida (un autómata celular) fue el glider, es decir una especie de patrón autoperpetuo que se mueve diagonalmente a través de una malla en un autómata celular.
\\

Estas entidades son importantes porque transmiten información relevante en ese mundo virtual y también hacen algunas cosas curiosas como formar objetos más complejos o jalar y empujar a otros objetos.
\\

Inclusive es posible hacer que dichas entidades sean ordenadas de tal manera que procesen información tal y como lo harían las compuertas lógicas, o de una manera más compleja lograr equivalentes a una máquina de Turing. De hecho, es posible lograr que los glider puedan computar.
\\

Un investigador llamado Tommasso Tolofi se dio cuenta que los gliders no son las únicas partículas existentes en el Juego de la Vida, sino que hay muchísimas más con propiedades distintas y que, cuando colisionan, pueden formar otras partículas fluyendo en distintas direcciones.
\\

Todo esto no es tan entendible hasta la parte en que se menciona que cada partícula no es más que una cadena de bits; y cuando una partícula interactúa con otra entonces puede terminar produciendo otra cadena de bits.
\\

El profesor Genaro Juárez M. y sus compañeros mencionan que los supercolisionadores pueden emular una enorme cantidad de colisiones naturales que de otra manera sería casi imposible de modelar.
\\

Este tipo de colisiones basadas en cálculos computacionales son un gran avance sobre el cómputo convencional porque también es posible compartir muchas propiedades en común con el sistema que se está emulando en ese momento.
\\

Realmente es algo muy interesante que seguramente en un futuro muy cercano se vea aplicándose a modelos de investigación mucho más precisos que requieran tal cantidad de detalle para resolver algunos problemas en sus diseños actuales.
}

\vspace{2cm}

\section*{Bibliograf\'ia}

\noindent \url{http://technologyreview.com/view/424096/computer-scientists-build-cellular-automaton-supercollider/}
\\

\large{\hfill \textbf{Hecho en } \LaTeX - \url{multiaportes.com}}

\end{document}